\chapter{F-Tract}\label{ch:ftract}

The entry point of this thesis is the paper Communication dynamics in the human connectome shape the cortex-wide propagation of direct electrical stimulation by Sequin et al., published in 2022. It shows that the network communication models discussed in Chapter \TODO[odkaz na kapitolu] computed using structural connectome based on DW-MRI can explain the propagation of focal electrical stimulation through the brain. 

The main question of this thesis is whether it is possible to generalize the methodology used by Seugin et al. to TMS-EEG data. Before we dive into that, it was necessary to confirm that we are able to replicate the results presented in the paper, which is the topic of this chapter. 

% The entry point to this topic will be the recent results on metrics based on network communication models capturing some of the relationship between structural connectivity and stimulus response. The original work, done on invasive intracranial data in a large cohort, serves as a basis for evaluating the extent to which these observations can be replicated in noninvasive stimulation and recordings. The student will engage with openly available datasets, namely the summary data of the Functional Brain Tractography project (F-TRACT), which provides the response probabilities and amplitudes between brain regions from intracranial recordings.

\section{iEEG data preprocessing}



\section{Results}

\TODO[probability, amplitude, delay jenom odmávat, že to je k ničemu]