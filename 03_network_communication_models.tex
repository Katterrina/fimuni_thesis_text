\chapter{Network communication models in neuroscience}

\TODO[najít ještě nějaký hezký zdroj, ať tam necituji pořád ten jeden článek]

Unveil the principles of communication and information processing in the brain is a primary goal of neuroscience. Network neuroscience views the brain as a complex network of anatomical connections. It tries to use tools rooted in graph theory to achieve this goal.  

Organization of connectomes follows a number of complex topological properties, such as modular and hierarchical structure or small-world\footnote{Small-world network combine high clustering (nodes tend to form densely connected clusters) and short characteristic path length (spatially distant nodes are, on average, connected via a small number of edges). \cite{seguin_brain_2023}} architecture. They are believed to evolve in support of efficient neural communication. \cite{seguin_brain_2023,avena-koenigsberger_communication_2018}

It is well known that structural connections allow direct communication between neural elements. However, the communication between brain regions that are not anatomically connected is not well understood. A growing number of communication models aim to provide insight into the communication process. The communication model is an algorithm \TODO \cite{seguin_brain_2023}

Currently used communication models were greatly summarized by Seguin, Sporns, and Zalesky in the review article \textit{Brain network communication: concepts, models and applications} published in 2023. \cite{seguin_brain_2023} We present the main ideas of network communication models in this chapter. \TODO[asi napsat, že popíšu jenom ty, které ]

\subsection{From decentralized to centralized communication models}

Following the methodology proposed in the paper, we organize the models into three groups: diffusion processes, parametric models, and routing protocols. The models are presented from decentralized to centralized. 

The degree of centralization is an important feature of a network communication model. \TODO[co to je a proč]

% various communication models on toy networks are placed along a spectrum of information, qualitatively representing how much (if any) information is needed for each communication process to take place.

\subsection{Diffusion processes}

\subsection{Parametric models}

\subsection{Routing protocols}