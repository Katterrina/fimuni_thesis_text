\chapter{Discussion}

We aim to explore the possibility of studying TMS-evoked potentials through complex network analysis. Specifically, we want to look at the problem from the point of view of network communication models applied to the structural connectome graph. 

The first step on our way is a confirmation of the results obtained by Seguin et al. \cite{seguin_communication_2023} for iEEG measurements -- correlations of response probability and amplitude with structural connectivity matrices and network communication models. We used publicly available versions of the F-TRACT dataset (50 ms and 200 ms response lengths). Let us recall that the data were measured in patients with drug-resistant epilepsy, which might influence the results. Even though the pathological activity was filtered out, the extent to which recordings from epilepsy patients could represent the brain activity of healthy subjects is still a topic of research. \cite{seguin_communication_2023}

We confirmed that the response probabilities and amplitudes from the F-Tract dataset correlate with the structural connectivity and network communication metrics across various settings. Among other parameters, we showed that the results are robust with respect to the structural connectivity dataset and group-averaging method, and in the case of probabilities also with respect to the brain parcellation. 

Our contribution in this part of the work is the use of structural connectivity lengths in the calculation of communication metrics. Seguin et al. \cite{seguin_communication_2023} worked only with structural connectivity weights. The lengths capture different aspect of the brain structure than weights. While the weights are higher with a higher number of streamlines between regions,\footnote{At least in this thesis. Generally, different options exist.} the lengths capture the real distance (not Euclidean distance) through the complex structure of the brain. We see in the results in Chapter \ref{ch:ftract} that the correlation of structural connectivity lengths with the response probability and amplitude is higher than for structural connectivity weights.

We also examined the correlations in the case of single ROI stimulation. Considering a single stimulated region (corresponding to one row in the T-TRACT matrix of probabilities) builds a connection between the F-TRACT dataset consisting of many pairs of stimulated and recording regions and our TMS-EEG data with a single stimulation target. However, this step is limited by the fact that F-TRACT does not contain the probabilities of activation for all pairs of stimulated and target regions. We faced this limitation during the search for the region in the Glasser parcellation used in the F-TRACT corresponding to the region stimulated in our TMS-EEG data. We overcame this obstacle by choosing a region that was the second-best fit according to the Dice score, but its row in the F-TRACT matrix contains more data. This compromise may cause some inaccuracies in the results.

Keeping in mind the limitations, the results obtained for F-TRACT serve as a basis for evaluating the extent to which these observations can be replicated in noninvasive stimulation and recordings. 

Moving to the TMS-EEG data, our first objective is to characterize the response to a stimulus by a vector, with each entry aligning with a specific ROI. This transformation shifts the focus from a spatiotemporal perspective of the original source reconstructed TMS-EEG data to solely spatial resolution. We proposed and tested several characteristics of the response, all of them including a threshold to filter out baseline activity. Binarized response, AUC, and the first peak time show significant correlations with the network communication metrics across various thresholds for empirical TMS-EEG data. On the other hand, the highest peak as a response characteristic does not show significant correlations with communication metrics, which is in contrast with amplitudes in the results for F-TRACT dataset. The reason for this may be that the TMS-EEG is an indirect stimulation and recording method, and it is necessary to reconstruct the data to obtain their possible sources, and because of that, it does not capture the peak heights well.

Besides the empirical TMS-EEG data, we repeated the analysis for simulated data by Momi et al. \cite{momi_tms-evoked_2023}. There is a link between the results in response characterization by first peak time; it correlates well with communication metrics for both empirical and simulated data. On the other hand, the area under the curve, which shows high correlations for empirical data, is much worse for simulated data. A possible reason for this is that the model used for data simulation models the first peak timing quite well because it is a simple characteristic but fails to model the overall complexity of the response captured by AUC in empirical data. This might be a suggestion for improvement of the model for artificial TMS-EEG data generation.

Comparing the results for TMS-EEG empirical data with F-TRACT, we see that the correlations are overall lower. That is not a surprise because of the indirect nature of TMS-EEG. More importantly, there are no partial correlations for TMS evoked response with structural connectivity and communication models when we control the influence of Euclidean distance, no matter how it is characterized. This is related to the fact that the correlation of TMS-evoked response is always higher with the Euclidean distance than with the communication metrics. In conclusion, although the TMS-evoked response correlates with the network communication metrics, Euclidean distance influences its character greatly.

In order to get better insight into the difference between response probability and its characterization in TMS-EEG, which may be the reason for the difference in correlations with communication metrics, we compare the response probabilities and characteristics with each other. Direct comparison is impossible because the data do not share a common parcellation. Because of that, we used the Dice score for mapping from Schaefer 200 (TMS-EEG) to Glasser (F-TRACT) parcellation. It resulted in a Spearman correlation coefficient $r\approx0.5$ between the response probabilities and AUC or first peak time. That suggests that there is a relationship between the probability and other response characteristics and its investigation may be a subject of further research. We also confirmed that the mapping is reasonable by checking the correlation of Euclidean distances in the two parcellations. However, we used a greedy approach, assigning the Schaefer ROI with the highest Dice score to each Glasser ROI, which left some Schaefer ROIs unused. This could definitely be improved in further work.

Our experiments are limited by several aspects of the used TMS-EEG data. First of all, we have only one dataset with only one stimulation target. We searched for other publicly available datasets with stimulation at the same site or elsewhere, but our search was unsuccessful. We found a paper by Fecchio et al. \cite{fecchio_spectral_2017} providing TMS-EEG data for the primary motor cortex and other cortical sites, but unfortunately the data are not source-reconstructed. EEG source reconstruction is above the scope of this thesis, but it might be a further direction of research.

Another limitation comes with the fact that the TMS-EEG data are group-averaged. It would be interesting to do the analysis on a subject level because group averaging of the time series could veil some aspects of the TMS-evoked response.

\TODO[možná něco o tom, že jsem všechny výsledky potvrdili s víc maticemi?]


