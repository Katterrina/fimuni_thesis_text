\chapter{Generalize F-TRACT approach to TMS-EEG}\label{ch:pytepfit}

\section{TMS-EEG data}\label{sec:reponse_definition}

There is a huge difference between the summarized iEEG functional data available in F-TRACT and the TMS-EEG data used in this section. 

The F-TRACT dataset provides probabilities. If we consider the stimulation of one ROI, there is a vector of probabilities that there are significant responses in the other ROIs. On the other hand, the TMS-EEG data used here are time series showing the reaction in time for each ROI after the target site stimulation. 

\subsection{Empirical data}

We used TMS-EEG data published by the Rogasch group\footnote{\url{https://figshare.com/articles/dataset/TEPs-_SEPs/7440713}} \cite{biabani_characterizing_2019} where high-density EEG was recorded following M1 stimulation in 20 healthy young individuals. TMS-EEG evoked potential (TEP) source reconstruction was performed by Momi et al. \cite{momi_tms-evoked_2023}, resulting in group-averaged time series, one time series per ROI in Schaefer200 parcellation. See Figure \ref{fig:tms-empirical-data}. 

\begin{figure}
    \centering
    \includegraphics[width=\textwidth]{images/nootebook_generated/pytepfit_results/empirical/200/not_over_threshold_nan/data.pdf}
    \caption[TMS-EEG empirical data]{TMS-EEG empirical data, stimulation at 100 ms after the start of the measurement. The TEPs are colored based on Yeo7 functional networks.}
    \label{fig:tms-empirical-data}
\end{figure}

\subsection{Simulated data}

Besides the source-reconstructed empirical TEPs, Momi et al. published simulated TEPs. The whole modeling process is described in their paper \textit{TMS-evoked responses are driven by recurrent large-scale network dynamics}. 

Briefly, the model consists of 200 brain regions (based on Schaefer200 parcellation) connected by weights of some anatomical connectivity matrix. The nodes represent the averaged activity of the specific brain region and activity in each node is modeled using a set of equations. Then, the model was fitted to the empirical data of each subject, and the resulting TEPs were averaged. \cite{deco_perturbation_2018, momi_tms-evoked_2023}

The resulting simulated data are plotted in Figure \ref{fig:tms-simulated-data}. We see that the TEPs are much smoother than the empirical in Figure \ref{fig:tms-empirical-data}.

\begin{figure}
    \centering
    \includegraphics[width=\textwidth]{images/nootebook_generated/pytepfit_results/simulated/200/not_over_threshold_nan/data.pdf}
    \caption[TMS-EEG simulated data]{TMS-EEG simulated data, stimulation at 100 ms after the start of the measurement. The TEPs are colored based on Yeo7 functional networks.}
    \label{fig:tms-simulated-data}
\end{figure}

\subsection{Response definition}

To follow the methodology used by Seugin et al. for the F-TRACT data, it is necessary to decide how to get one number from the time series of each ROI and, followingly, a vector characterizing the response in the whole brain. We can not use the probability of the response because the TEPs were measured only in 20 subjects and we have the group-averaged data.

This section discusses all the options used later in the work.\footnote{We considered a few more, but we did not include them because they are very similar to the ones in the list.} They are also visualized in Figure \ref{fig:tms-respondse-definition}.

\begin{itemize}
    \item \textbf{binary (01) response} If the TEP exceeds a predefined threshold, the value is set to 1, otherwise it is set to 0.
    \item \textbf{first/highest peak} The weight of the response is defined as the height of the first/highest peak above a predefined threshold. If the TEP does not exceed the threshold, it is set to \texttt{nan} and is not considered in the analysis.\footnote{Alternatively, we can replace all \texttt{nan}s in the definitions with zeros and include them in the analysis. The approaches capture different properties of the response.} The highest peak corresponds to amplitude in F-TRACT. 
    \item \textbf{first/highest peak time} The weight of the response is defined as the time when the first/highest peak above a predefined threshold occurs. If the TEP does not exceed the threshold, it is set to \texttt{nan} and is not considered in the analysis. The idea behind this definition is that we assume the response to propagate faster to regions that are \uv{more} connected (for example, there are edges with higher weights or shorter lengths in structural connectome).
    \item \textbf{AUC} The weight of the response is defined as the area under the curve if the curve exceeds a predefined threshold, \texttt{nan} otherwise. The key idea of this definition is that AUC captures the overall \uv{power} of the response. It is systematically different from the other definitions, so it might reveal different relationships.
\end{itemize}

\begin{figure}
    \centering
    \includegraphics[width=\textwidth]{images/nootebook_generated/pytepfit_results/simulated/200/not_over_threshold_nan/7Networks_LH_Cont_Cing_2-lh_response_def.pdf}
    \caption[TMS-EEG response definitions -- illustration]{TMS-EEG response definitions - illustration with one ROI, 200 ms response (stimulation at 0).}
    \label{fig:tms-respondse-definition}
\end{figure}

\subsubsection{Threshold selection}

All the response definitions work with some threshold. There is always a baseline activity in the brain, as shown in Figure \ref{fig:tms-empirical-data} before the stimulation. We need to filter out the activity after the stimulation caused by this natural process and keep only responses truly evoked by the TMS stimulation. However, there is nothing like \uv{the right threshold} and it depends on the specific data used in each case. Because of that, we tried several thresholds, the lowest corresponding to the highest peak in the activity before the stimulation and the highest selected such that there are at least 30 TEPs over that threshold.

\subsubsection{Response length}

Besides the threshold, it is important to select the length of the response we want to consider. The TMS coil makes a sound when turned on, so the later responses include the reaction of the brain to the auditory stimulus. On the other hand, as shown in Section \ref{sec:response-length_F-Tract} the earlier responses might be driven simply by the Euclidean distance. We tried several response lengths, including 50 ms and 200 ms, which gave us results comparable to those obtained using F-TRACT.

\section{Results for empirical data}

In this section, we present the results showing that, for some of the response definitions (specifically binarized TEP, AUC, and first peak height) there are significant correlations across various thresholds between the response and structural connectivity and the communication metrics derived from it. 

For the structural connectivity, we used the Mica-Mics dataset with Rosen and Halgrens's preprocessing method because it gave us the best results for the F-TRACT dataset (Section \ref{sec:sc-robustness_ftract}) and it also provides the Schaefer200 parcellation used here, which allows better comparison of the results. We used the Spearman correlation coefficient, results are considered significant for $p<0.05$.

\subsection{Results by response definition -- binary}

First of all, if the response is defined as binary, it correlates with Euclidean distance across response lengths 50, 100, 150, 200, and 300 milliseconds and various thresholds ($r \approx 0.4$ depending on the parameters). It serves as a sanity check that the calculations are correct because Sections \ref{sec:ftract_results} and \ref{sec:ftract_results_per_roi} show that there is a high correlation between the response probability and amplitude and Euclidean distance. 

We show the results for 200 ms response across various thresholds in Figure \ref{fig:tms_01_200}. The shortest path efficiency and the navigation efficiency correlations are similar to the ones achieved with Euclidean distance, which is similar to the results with F-TRACT in Figure \ref{fig:ftract_alldata_long_probabilities}.

\begin{figure}
    \centering
    \includegraphics[width=\textwidth]{images/nootebook_generated/pytepfit_results/empirical/200/not_over_threshold_nan/Response defined as 01-response.pdf}
    \caption[Binarized TEP (200 ms) correlations]{Spearman correlation coefficient of binarized empirical TEP (200 ms response) with structural connectivity and communication metrics. Darker color denoted a higher absolute value of the correlation; missing values are non-significant ($p>0.05$).}
    \label{fig:tms_01_200}
\end{figure}

\subsection{Results by response definition -- AUC}

Based on our results, AUC of TEP above a threshold correlates best with the structural connectivity and communication metrics.

The results for 50 ms response length (Figure \ref{fig:tms_auc_50}) differ from longer responses ($\geq100$ ms) because there are significant correlations consistently expressed only for Euclidean distance, shortest path efficiency, and navigation efficiency (and only for lower thresholds). For the longer responses (represented by Figure \ref{fig:tms_auc_200} for 200 ms), the correlations are overall higher and appear across various thresholds. 

Nevertheless, the response AUC correlation is consistently highest with Euclidean distance, the shortest path efficiency and navigation efficiency, followed by structural connectivity lengths, search information and communicability, which corresponds to the results in F-TRACT considering all pairs of ROIs (Figure \ref{fig:ftract_mica_short_probabilities} for 50 ms response and Figure \ref{fig:ftract_mica_long_probabilities} for 200 ms response).

\begin{figure}
    \centering
    \includegraphics[width=\textwidth]{images/nootebook_generated/pytepfit_results/empirical/50/not_over_threshold_nan/Response defined as AUC.pdf}
    \caption[TEPs AUC (50 ms) correlation with SC and communication metrics]{Spearman correlation coefficient of AUC of empirical TEP (50 ms response) with structural connectivity and communication metrics. Darker color denoted a higher absolute value of the correlation; missing values are non-significant ($p>0.05$).}
    \label{fig:tms_auc_50}
\end{figure}

\begin{figure}
    \centering
    \includegraphics[width=\textwidth]{images/nootebook_generated/pytepfit_results/empirical/200/not_over_threshold_nan/Response defined as AUC.pdf}
    \caption[TEPs AUC (200 ms) correlations]{Spearman correlation coefficient of AUC of empirical TEP (200 ms response) with structural connectivity and communication metrics. Darker color denoted a higher absolute value of the correlation; missing values are non-significant ($p>0.05$).}
    \label{fig:tms_auc_200}
\end{figure}

\subsection{Results by response definition -- first peak}

If we define the response strength as the height of the first peak, we need to consider an interval long enough so the peaks can occur. Looking at Figure \ref{fig:tms-empirical-data}, it could be seen that 50 ms is insufficient. Based on our experiments, we should take at least 150 ms response. 

Taking into account only responses longer than 150 ms, the first peak height correlates with Euclidean distance, shortest path efficiency, navigation efficiency, and communicability across all the thresholds. 

\TODO[zmínit, že tady SPEW vychází víc než SPE?]
%An interesting observation is that unlike all the other cases so far, here the shortest path efficiency calculated using structural connectivity weights returns higher correlations than the shortest path efficiency calculated using structural connectivity weights. It might be a coincidence, but 

\begin{figure}
    \centering
    \includegraphics[width=\textwidth]{images/nootebook_generated/pytepfit_results/empirical/200/not_over_threshold_nan/Response defined as first_peak.pdf}
    \caption[TEPs first peak (200 ms) correlations]{Spearman correlation coefficient of the height of the first peak in empirical TEP (200 ms response) with structural connectivity and communication metrics. Darker color denoted a higher absolute value of the correlation; missing values are non-significant ($p>0.05$).}
    \label{fig:tms_first_200}
\end{figure}

\subsection{Results by response definition -- higest peak}

It is surprising that there are only a few significant correlations across the response lengths 50, 100, 200, and 300 milliseconds and various thresholds when we use the highest peak as a response characterization. It seems that there is no systematic relationship between the highest peak height and structural connectivity, and the few significant correlations are just coincidences. It stands in contrast with the results obtained for response amplitudes from F-TRACT in Figure \ref{fig:ftract_alldata_long_amplitudes}, where we see significant correlations of the amplitudes with structural connectivity and communication metrics.

There is an exception for a response length of 150 ms, which shows many significant correlations. It is probably because the highest peaks correspond to the first peaks for this response length; the plot for this setting looks very similar to Figure \ref{fig:tms_first_200}. 

\subsection{Results by response definition -- peak times}

Defining the response as the first peak time proved to be a dead end, as the results are largely non-significant. Some significant correlations occur for the characterization of the response by highest peak time, but they are lower than for AUC and first peak height.

\subsection{Response length}

The results are changing a lot with response length. As mentioned above, if the response is characterized by the first peak height, the length should be at least around 150 ms. If the response is shorter, not all peaks occur in the time. 

If we characterize the response using AUC, the results for a really short response length (50 ms) are different from the other response lengths ($\geq100$ ms), as shown in Figure \ref{fig:tms_auc_50} and Figure \ref{fig:tms_auc_200}.

\section{Empirical vs simulated data}

The main question we aim to answer by comparing results for empirical and simulated data is if the simulated data express similar behaviour to the empirical ones. We can see just by looking at the Figures \ref{fig:tms-empirical-data} and \ref{fig:tms-simulated-data} that the distribution, height and timing of the peaks and AUC are different. 

Does it mean that the data act differently with respect to the analysis of the response correlation with structural connectivity and communication metrics? Possible similarity of the results could mean that the simulated data grasps some aspects of the empirical data and we can study the simulated data to reveal some aspects of the nature of the brain.

\subsection{Similarities}

First important similarity lies in results for the binarized response. It shows the same pattern as we have seen before, the response correlates with Euclidean distance, the shortest path efficiency and the navigation efficiency. Figure \ref{fig:tms_binary_100_simulated} shows that for 100 ms response. 

\begin{figure}
    \centering
    \includegraphics[height=\textwidth]{images/nootebook_generated/pytepfit_results/simulated/100/not_over_threshold_nan/Response defined as 01-response.pdf}
    \caption[Binarized TEP (100 ms) correlations (simualted data)]{Spearman correlation coefficient of binarized empirical TEP (100 ms response) with structural connectivity and communication metrics. Darker color denoted a higher absolute value of the correlation; missing values are non-significant ($p>0.05$).}
    \label{fig:tms_binary_100_simulated}
\end{figure}

Another similarity can be found for response characterized by first peak height. Even though the correlations are overall higher for simulated data, Figure \ref{fig:tms_first_200_simulated} shows similar pattern to Figure \ref{fig:tms_first_200} with the highest correlations of the first peak heights with Euclidean distance, shortest path efficiency and navigation efficiency across all thresholds. 

\begin{figure}
    \centering
    \includegraphics[height=\textwidth]{images/nootebook_generated/pytepfit_results/simulated/200/not_over_threshold_nan/Response defined as first_peak.pdf}
    \caption[TEPs first peak (200 ms) correlations (simulated data)]{Spearman correlation coefficient of the height of the first peak in simulated TEP (200 ms response) with structural connectivity and communication metrics. Darker color denoted a higher absolute value of the correlation; missing values are non-significant ($p>0.05$).}
    \label{fig:tms_first_200_simulated}
\end{figure}

\subsection{Differences}

Unsurprisingly, there are many differences in the results calculated using simulated and empirical data. First of all, considering response length at least 100 ms, the highest correlations of the response with communication metrics are obtained for response strength defined as its highest peak height. That is probably because the simulated data, unlike the empirical data, does not contain noise later after the stimulation. The highest peak is also often the first one.

Response characterized by AUC correlates poorly with structural connectivity and communication metrics if it is calculated using the simulated data. On the other hand, the timing of the peaks correlates better with communication metrics for the simulated data than for the empirical data. 

In conclusion, the similarities and differences seem to follow the difference in data simulation and generation through the natural process in the brain. The simulated data capture the peaks which might be noisy in the empirical data. On the other hand, the simulated data do not model the overall complexity of the response, captured by AUC in the epirical data, very well.