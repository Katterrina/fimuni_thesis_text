\chapter{Implementation of the analysis}

There is a repository in the attachments with all the codes to perform the analysis in the thesis. It consists of Juypetr notebooks that include the individual steps and their descriptions. The repository structure is described in Figure \ref{fig:repo-structure} for better orientation. 

\begin{figure}[!h]
\centering
\begin{minipage}{0.9\textwidth}
\dirtree{%
.0 .
.1 data.
.2 external.
.3 README.md \textit{(information about external data sources)}.
.3 \ldots.
.2 interim \textit{(data generated using the notebooks)}.
.1 figures \textit{(figures generated using the notebooks)}.
.2 ftract\_results. 
.2 ftract\_results\_per\_roi.
.2 tmseeg\_resluts.
.2 tmseeg\_ftract\_comparison\_resluts.
.2 tmseeg\_ftract\_response\_comparison.
.2 sc\_comparison.
.1 notebooks.
.2 00\_structural\_connectivity.
.2 01\_f-tract.
.2 02\_pytepfit \textit{(notebooks related to TMS-EEG data)}.
.2 03\_compare\_f-tract\_pytepfit.
.2 XX\_centroids\_and\_parcellations.
.1 src \textit{(data loaders, plotting functions etc.)}.
}
    \end{minipage}
    \caption{Repository structure}
    \label{fig:repo-structure}
\end{figure}

The repository also contains all the figures generated during our work in the \texttt{figures} folder. The folder's subdirectories contain the figures organized based on the specific parameters used to generate them, for example, the parcellation or the minimal number of streamlines to consider an edge between two regions, response length (in the case of TMS-EEG), and others. We provide a list of the parameters with their possible values below.

\newpage

\begin{itemize}
    \item parcellation
    \begin{itemize}
        \item \texttt{DKT} (Deskian-Killiany parcellation)
        \item \texttt{MNI-HCP-MMP1} (Glasser parcellation)
        \item \texttt{schaefer} (Schaefer200 parcellation)
    \end{itemize}
    \item response length
    \begin{itemize}
        \item \texttt{long|short} indicating 50~ms or 200~ms responses, used for F-Trcat results
        \item \texttt{XXXms} indicating the response length is \texttt{XXX} milliseconds, used for TMS-EEG results
    \end{itemize}
    \item minimal number of streamlines, see Section \ref{sec:min_streamlines}
    \begin{itemize}
        \item \texttt{min\_X\_streamlines}
    \end{itemize}
    \item connectivity matrices density, see Section \ref{sec:density}
    \begin{itemize}
        \item \texttt{X\_density} 
    \end{itemize}
    \item close ROI pairs excluded, mentioned in Section \ref{sec:f-tract_data_description} (applied only for F-Tract dataset)
    \begin{itemize}
        \item \texttt{EDX} ROIs closer than \texttt{X} mm excluded from the analysis 
    \end{itemize}
    \item the usage of empirical or simulated data (for TMS-EEG)
    \begin{itemize}
        \item \texttt{empirical|simulated} 
    \end{itemize} 
    \item response characteristic in TMS-EEG (for TMS-EEG and F-Tract response comparison)
    \begin{itemize}
        \item \texttt{AUC} 
        \item \texttt{FP} first peak latency 
    \end{itemize} 
    \item how to handle response for ROI when it does not exceed a threshold in TMS-EEG, see Section \ref{sec:reponse_definition} (for TMS-EEG)
    \begin{itemize}
        \item \texttt{not\_over\_threshold\_[0|nan]}, if \texttt{nan}, the region is not included in the analysis
    \end{itemize} 
\end{itemize}