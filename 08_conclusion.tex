\chapter*{Conclusion}
%% Unlike \chapter, \chapter* does not update the headings and does not
%% enter the chapter into the table of contents. If we want correct
%% headings and a table of contents entry, we must add them manually:
\markright{\textsc{Conclusion}}
\addcontentsline{toc}{chapter}{Conclusion}

The goal of this thesis was to evaluate the spatiotemporal patterns of the human brain activity evoked by direct stimulation using the methods drawn from the field of complex network analysis. Specifically, we want to look at the problem from the point of view of network communication models applied to the structural connectome graph. 

The first step on our way is the replication of the results obtained by Seguin et al. \cite{seguin_communication_2023} for iEEG measurements -- correlations of response probability and amplitude with structural connectivity matrices and network communication models. We used publicly available versions of the F-Tract dataset (50 ms and 200 ms response lengths).

We confirmed that the response probabilities and amplitudes from the F-Tract dataset correlate with the structural connectivity and network communication metrics across various settings. Among other parameters, we showed that the results are robust with respect to the structural connectivity dataset and group-averaging method, and in the case of probabilities also with respect to the brain parcellation. 

Our contribution in this part of the work is the use of structural connectivity lengths in the calculation of communication metrics. Seguin et al. \cite{seguin_communication_2023} worked only with structural connectivity weights. The lengths capture different aspect of the brain structure than weights. While the weights scale with the number of streamlines between regions,\footnote{At least in this thesis. Generally, different options exist. \cite{zhang_quantitative_2022}} the lengths capture the distance the signal has to cover between two nodes (not Euclidean distance) through the complex structure of the brain, and is proportional to the communication cost. The results in Chapter \ref{ch:ftract} show that the correlation of structural connectivity lengths with the response probability and amplitude is higher than for structural connectivity weights, but the lengths still explain less variability in the response probability than the network metrics based on communication models.

We also examined the correlations in the case of single ROI stimulation. Considering a single stimulated region (corresponding to one row in the F-Tract matrix of probabilities) builds a connection between the F-Tract dataset consisting of many pairs of stimulated and recording regions and our TMS-EEG data with a single stimulation target. However, this step is limited by the fact that the F-Tract dataset does not contain the probabilities of activation for all pairs of stimulated and target regions. We faced this limitation during the search for the region in the Glasser parcellation used in the F-Tract corresponding to the region stimulated in our TMS-EEG data. We overcame this obstacle by choosing a region that was the second-best fit according to the Dice score, but its row in the F-Tract matrix contains more data. This compromise may cause some inaccuracies in the results.

Keeping in mind the limitations, the results obtained for F-Tract serve as a basis for evaluating the extent to which these observations can be replicated in noninvasive stimulation and recordings. 

Moving to the TMS-EEG data, we proposed and tested several characteristics of the response, all of them including a threshold to filter out baseline activity. Binarized response, AUC, and the first peak latency show significant correlations with the network communication metrics across various thresholds for empirical TMS-EEG data. On the other hand, the highest peak as a response characteristic does not show significant correlations with communication metrics, which is in contrast with amplitudes in the results for the F-Tract dataset. One of the possible reasons for this mismatch might lie in the limitations of the source reconstruction applied to the EEG scalp signals, which is necessary to obtain the estimate of activity in the space of the brain. Because of that, it does not capture the peak heights well.

Besides the empirical TMS-EEG data, we repeated the analysis for simulated data by Momi et al. \cite{momi_tms-evoked_2023}. There is a link between the results in response characterization by first peak latency; it correlates well with communication metrics for both empirical and simulated data. On the other hand, the area under the curve, which shows high correlations for empirical data, is much worse for simulated data. A possible reason for this is that the model used for data simulation models the first peak timing quite well because it is a simple characteristic but fails to model the overall complexity of the response captured by AUC in empirical data. This might be a suggestion for improvement of the model for artificial TMS-EEG data generation.

Comparing the results for TMS-EEG empirical data with F-Tract, we see that the correlations with the communication models and the structural connectomes are overall lower. That is not a surprise because of the indirect nature of TMS-EEG. More importantly, there are no partial correlations for TMS evoked response with structural connectivity and communication models when we control the influence of Euclidean distance, no matter how it is characterized. This is related to the fact that the correlation of TMS-evoked response is always higher with the Euclidean distance than with the communication metrics. In conclusion, although the TMS-evoked response correlates with the network communication metrics, Euclidean distance influences its character greatly.

In order to get better insight into the difference between response probability and its characterization in TMS-EEG, which may be the reason for the difference in correlations with communication metrics, we compare the response probabilities and characteristics with each other. Direct comparison is impossible with the publicly available data because the datasets do not share a common parcellation. Because of that, we used the Dice score for mapping from Schaefer 200 (TMS-EEG) to Glasser (F-Tract) parcellation. It resulted in a Spearman correlation coefficient $r\approx0.5$ between the response probabilities and AUC or first peak latency. That suggests that there is a relationship between the probability and other response characteristics and its investigation may be a subject of further research. We also confirmed that the mapping is reasonable by checking the correlation of Euclidean distances in the two parcellations. However, we used a greedy approach, assigning the Schaefer ROI with the highest Dice score to each Glasser ROI, which left some Schaefer ROIs unused. 

Our experiments are limited by several aspects of the used TMS-EEG data. First of all, we have only one dataset with only one stimulation target. We searched for other publicly available datasets with stimulation at the same site or elsewhere, but our search was unsuccessful. We found a paper by Fecchio et al. \cite{fecchio_spectral_2017} providing TMS-EEG data for the primary motor cortex and other cortical sites, but unfortunately the data are not source-reconstructed. EEG source reconstruction is beyond the scope of this thesis, but it might be a further direction of research.

Another limitation comes with the fact that the TMS-EEG data are group-averaged for the purposes of this thesis. It would be interesting to do the analysis on a subject level because group averaging of the time series could veil some aspects of the TMS-evoked response.

We performed a robustness analysis for all the results. We confirmed that the correlations are not dependent on specific structural connectivity datasets. Regarding the group averaging method, it seems to us that Rosen and Halgren's method is the best choice in this case. However, the differences are small and further analysis would be needed to confirm that. We also tried to use different parcellations for the F-Tract dataset, which shows the importance of parcellation selection.

And what about the question raised in Chapter \ref{ch:networks}, asking if the centralized models explain communication within the brain better than decentralized ones? Our results show the highest correlation of the response with the most common centralized communication models, shortest path efficiency, for both F-Tract and TMS-EEG data. It is followed by navigation efficiency, which is a centralized metric as well. However, the success of navigation efficiency should be interpreted carefully, as it utilizes Euclidean distance, which itself correlates with the responses very well.

\section*{Future work}

To conclude the work, let us summarize possible directions for future work. It would be great to try to analyze source-reconstructed TMS data for other ROIs, especially if they were in parcellation compatible with FTRACT, allowing direct comparison of the results.

Another direction would be a detailed analysis of the late response complexity discrepancies of the empirical and simulated TMS-EEG data. Such analysis would be beneficial for further improvement of the model generating the simulations.

Last but not least, it would be interesting to leverage the advantage of the noninvasive nature of TMS-EEG to study inter-subject variability of the TMS-evoked responses, which is not possible for the intracranial data in F-Tract.