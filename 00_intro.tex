\chapter*{Introduction}
%% Unlike \chapter, \chapter* does not update the headings and does not
%% enter the chapter into the table of contents. If we want correct
%% headings and a table of contents entry, we must add them manually:
\markright{\textsc{Introduction}}
\addcontentsline{toc}{chapter}{Introduction}

There is a long history of neuroscientific studies viewing the nervous system as a complex network.\cite{sporns_structure_2013} This thesis aims to explore methodologies rooted in complex network analysis to study both empirical and simulated electroencephalography (EEG) recordings of brain responses to transcranial magnetic stimulation. 

We chose the recent paper \textit{Communication dynamics in the human connectome shape the cortex-wide propagation of direct electrical stimulation} by Sequin et al. \cite{seguin_communication_2023} as an entry point to this topic. Seguin et al. present results capturing some of the relationships between structural connectivity and stimulus response based on network communication models. Their work, done on invasive intracranial data in a large cohort, serves as a guide for the application of the network communication models methodology to noninvasive data.

This thesis is organized as follows. Chapter \ref{ch:brain} provides the necessary information about the biology of the human brain and technical aspects of the measuring techniques employed in data acquisition. This is immediately followed by a description of the network construction from the measurements of the brain in Chapter \ref{ch:SC}.

Chapter \ref{ch:networks} is devoted to the description of the network communication models. It includes their mathematical formulations, and it relates them to the biological aspects of communication processes in the brain.

Chapter \ref{ch:ftract} includes a replication of the results by Sequin et al. and our extension of their work. They focused on whole-brain stimulus-response probabilities, but we also examined the case considering only one stimulated region. 

The next chapters are dedicated to the application of the methodology of Sequin et al. to both empirical and simulated noninvasive electroencephalography (EEG) recordings of brain responses to transcranial magnetic stimulation and the comparison of obtained results with the original work on intracranial data. Chapter \ref{ch:pytepfit} discusses several possibilities for how to prepare the TMS-EEG data for the application of the analysis by Sequin et al. and our results of the application. It also compares the results for empirical and simulated TMS-evoked potentials. Followingly, Chapter \ref{ch:compare} compares the results for intracranial and transcranial data.

Last but not least, Chapter \ref{ch:SC_indepth} focuses on a robustness analysis of the results, especially regarding variations in structural connectivity datasets and the methods of group-averaging structural connectivity data from several subjects.


