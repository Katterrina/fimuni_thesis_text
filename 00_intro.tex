\chapter*{Introduction}
%% Unlike \chapter, \chapter* does not update the headings and does not
%% enter the chapter into the table of contents. If we want correct
%% headings and a table of contents entry, we must add them manually:
\markright{\textsc{Introduction}}
\addcontentsline{toc}{chapter}{Introduction}

There is a long history of neuroscientific studies viewing the nervous system as a complex network.\cite{sporns_structure_2013} \TODO[dokončit]

\TODO[Pokus s ChatGPT, přepsat, uloženo pro inspiraci v angličtině] 

Understanding the spatio-temporal patterns of brain activity evoked by direct stimulation is a critical area of research in neuroscience, with profound implications for both basic science and clinical applications. This thesis aims to explore methodologies rooted in complex network analysis to assess these patterns, specifically focusing on their applicability in both empirical and simulated electroencephalography (EEG) recordings of brain responses to stimulation.

Recent advancements in brain network analysis, particularly in metrics based on network communication models, have significantly enhanced our understanding of the relationships between structural connectivity and stimulus response. These models are designed to capture the intricate dynamics of how different brain regions communicate and respond to external stimuli. Notably, much of this progress has been achieved through the use of invasive intracranial data, which provides detailed insights into brain activity but is limited in its applicability due to its invasive nature.

This thesis builds upon the foundation of invasive studies by evaluating the extent to which these observations can be replicated using noninvasive stimulation and recording techniques. To achieve this, the thesis utilizes openly available datasets, including summary data from the Functional Brain Tractography project (F-TRACT). F-TRACT provides valuable insights into response probabilities and amplitudes between brain regions derived from intracranial recordings. Additionally, transcranial magnetic stimulation (TMS) datasets available through EBRAINS and other data-sharing platforms will be employed. These resources enable a comprehensive analysis of the brain’s response to direct stimulation using noninvasive methods.

The core of this research involves a detailed engagement with these datasets and the network communication models used to characterize the complexity of stimulus responses. The primary focus will be on the implementation and modification of these models, followed by their iterative application to the noninvasive data. This approach will allow for a thorough assessment of any discrepancies between the results obtained from noninvasive and invasive datasets. Given the complexity and depth of the subject matter, collaboration with experts at CEITEC MU will be essential. These experts will provide guidance on various aspects of the data and the broader neuroscientific context, ensuring the robustness and validity of the findings.

By integrating complex network analysis methodologies with empirical and simulated EEG data, this thesis seeks to advance our understanding of brain activity patterns in response to direct stimulation. The outcomes of this research could have significant implications for the development of noninvasive brain stimulation techniques and their applications in both research and clinical settings.


