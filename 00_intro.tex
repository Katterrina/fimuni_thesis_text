\chapter*{Introduction}
%% Unlike \chapter, \chapter* does not update the headings and does not
%% enter the chapter into the table of contents. If we want correct
%% headings and a table of contents entry, we must add them manually:
\markright{\textsc{Introduction}}
\addcontentsline{toc}{chapter}{Introduction}

There is a long history of neuroscientific studies viewing the nervous system as a complex network.\cite{sporns_structure_nodate} \TODO[dokončit]

%% \section{Obsah TODO}
%% \begin{markdown*}
%% # Obsah?
%% 
%% * Úvod
%% * Parcelace??? (někam musí přijít a nevím, kam)
%% * Strukturní konektivita
%%     * DW-MRI, tractography
%%     * jak se ze streamline counts (nebo tak něčeho) získá matice strukturní konektivity
%%         * consensus 
%%         * distance-dependent consensus
%%     * popsat konkrétní strukturní matice a jejich porovnání
%%         * Enigma
%%         * PyTepFit
%%         * Domhof
%%         * Rosen-Halgern (to je ta, co dobře koreluje s F-Tractem)
%% * Odezva na stimulaci
%%     * velmi krátce o fungování mozku (vedení signálu a tak)
%%     * EEG
%%         * krátce jak to funguje
%%         * rekonstrukce do objemu (alespoň zmínit)
%%     * elektrody v mozku vs TSM - co je stejné a co se liší
%% * Grafové vlastnosti SC pro šíření signálu
%%     * článek Sporns a spol - vysvětlení jednotlivých metrik a proč by měly dávat smysl
%% * F-Tract
%%     * co dělali
%%     * popis dat
%%     * co jsem dělala já
%%     * jak to vycházelo a proč se to může lišit
%% * TMS-EEG
%%     * co jsem dělala
%%     * jak to vycházelo a proč to nevycházelo
%% * diskuze
%% * závěr
%% 
%% \end{markdown*}
